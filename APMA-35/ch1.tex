\subs{Basic Mathematical Models; Direction Fields}
\begin{itemize}

\f
A \B{mathematical model} is a differential equation that describes some physical process. E.g.
\begin{align*}
	F &= ma = m\frac{dv}{dt}\\
	F &= mg - \gamma v \text{ where $\gamma = $ drag coefficient}\\
	\intertext{so putting the two equations together...}
	m\frac{dv}{dt} &= mg - \gamma v
\end{align*}

Constructing mathematical models involves recognizing and solving a few steps:
\begin{enumerate}
\f Identify independent and dependent variables. Keep in mind time is usually independent.
\f Choose units of measurement
\f Find a basic principal that governs the problem. For example: the rate of change of temperature of a glass of water is proportional to the difference between its temp and the environment's.
\f Express part 3 in terms of the variables in part 1. Make sure units match up. This can sometimes be a system of equations.
\end{enumerate}

\f \B{Direction fields} are formed by evaluating $\frac{dy}{dt} = f(t,y)$ at points in the $y-t$ plane. At each coordinate, a line segment is drawn with the slope of $f(t,y)$. 
\end{itemize}

\subs{Solutions of Some Differential Equations}
\begin{itemize}
\f The \B{initial value problem} is the solution of the differential equation, together with an initial condition
\f An example of a simple initial value problem
\begin{align*}
	\frac{dy}{dt} &= ay - b \text{ with } y(0) = y_0\\
	\frac{dy}{ay - b} &= dt\\
	\int \! \frac{dy}{ay - b} &= \int \! dt\\
	\frac{\log{ay-b}}{a} &= t\\
	e^{\log{ay-b}} &= e^{at}\\
	y &= \frac{b + e^{at}}{a}\\
	y &= \frac{b}{a} + Ce^{at}\\
	\intertext{plug in for the initial condition}
	y(0) = y_0 &= \frac{b}{a} + Ce^{0}\\
	y_0 &= \frac{b}{a} + C\\
	C &= y_0 - \frac{b}{a}\\
	\intertext{so our final equation looks like...}
	y &= \frac{b}{a} + (y_0 - \frac{b}{a})e^{at}\\
\end{align*}
\end{itemize}

\subs{Classification of Differential Equations}
\begin{itemize}
\f \B{Ordinary and Partial Differential Equations}
	\begin{itemize}
	\f Ordinary differential equations depend only on a single independent variable, and only \emph{ordinary derivatives} appear in the differential equation.
	\f Partial differential equations depend on multiple independent variables. Therefore \emph{partial derivatives} appear in them.
	\end{itemize}
\f \B{Systems of Differential Equations}. 
	This one is easy. If there are multiple unknown functions involved, you are dealing with a system of DEs. An example is the Lotka-Volterra (predator-prey) equations.
\f \B{Order of the Differential Equation}. The order of a differential equation is the order of the highest derivative that appears in the equation. Thus
$F(t, y(t), y'(t), y''(t), \ldots , y^n(t)) $
is order $n$. We look at functions like these in the form 
	\begin{equation} y^n = f(t, y, y', \ldots, y^{n-1})
	\end{equation}
\f \B{Linear and Nonlinear Equations}. 
	\begin{itemize}
	\f Linear equations are of the form: 
		$$ F(t, y, y', y'',\ldots , y^n) = 0$$
		The equation is a linear combination of all the derivatives up to $n$. Any $y$ term can be multiplied by any function of $t$ still, however. 
	\f Nonlinear equations, on the other hand, do not satisfy this constraint. Any equation with a product $yy'$, $y^2$ or a term like $\sin (y)$ are all nonlinear. 
	\end{itemize}
\end{itemize}

\subs{Historical Remarks}
Newton is great. Leibniz is also cool. Euler beats them both. The Bernoulli brothers loved their integrals, and a lot of fun was had by all.