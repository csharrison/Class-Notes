This chapter deals with differential equations of the first order:
\begin{equation}
\frac{dy}{dt} = f(t,y)
\end{equation}
Any differentiable function $y = \phi(t)$ that satisfies this equation for all $t$ in some interval is called a solution.

\subs{Linear Equations; Method of Integrating Factors}
The \B{first order linear equation} can be written in the form:
\begin{equation}
\frac{dy}{dt} + p(t)y = g(t)
\end{equation}
The method in part 1 cannot solve this generally unless $p(t)$ and $g(t)$ are constants. We need a new method.
\begin{itemize}
\f Find an integrating factor $\mu$, and multiply it on both sides: 
	$$\mu(t)\frac{dy}{dt} + \mu(t)p(t)y = \mu(t)g(t)$$
	
\f We now choose $\mu$ such that the left hand side is the derivative of some expression. Recall the product rule:
	$$\frac{d}{dt}\left[\mu(t)y\right] = \mu(t)\frac{dy}{dt} + \frac{d\mu(t)}{dt}y$$
\f Now we can just pick a $\mu(t)$ such that 
	\begin{align*} 
		\frac{d\mu(t)}{dt} &= \mu(t)p(t)\\
		\frac{d\mu(t)}{\mu(t)} &= p(t)dt\\
		\int \! \frac{d\mu(t)}{\mu(t)} &= \int \! p(t)dt\\
		\log{\mu(t)} &= \int \! p(t)dt\\
		\mu(t) &= e^{\int \! p(t)dt}
	\end{align*}

\f Now our equation looks something like this:
$$(\mu(t)y)' = \mu(t)g(t) \rightarrow (e^{\int \! p(t)dt)}y)' = e^{\int \! pt(1)dt}g(t) $$
\f Solving the equation yields 
$$y = \frac{\int \! \mu(t)g(t)dt + C}{\mu(t)}$$
\end{itemize}

\subs{Separable Equations}
Recall the general first order equation (2) is 
$$ \frac{dy}{dx} = f(x,y)$$
Note: this is not necessarily linear, and there is no universal method for solving this class of equations. We will consider a subclass of equations in this set. These equations are of the form
\begin{equation}
	M(x,y) + N(x,y)\frac{dy}{dx} = 0
\end{equation}
Note that we are simply rewriting equation (2). It is \emph{always} possible to do this, simply by setting $M(x,y) = -f(x,y)$ and $N(x,y) = 1$. \\
An equation is \B{separable} if $M$ is a function of $x$ only, and $N$ is a function of $y$ only. Thus,
$$M(x) + N(y)\frac{dy}{dx} = 0$$
\begin{equation}
M(x)dx + N(y)dy = 0
\end{equation}
This can be solved by integrating $M$ and $N$ to find an implicit solution.
$$
\int \! M(x)dx + \int \! N(y)dy = 0
$$
\B{Homogeneous Equations}. If the right side of $\frac{dy}{dx} = f(x,y)$ can be expressed as a ratio of $\frac{y}{x}$ only, the equation is said to be homogeneous. These can always be solved with a change of dependent variable.
\begin{align*}
\frac{dy}{dx} &= \frac{y - 4x}{x - y}\\
	&= \frac{y - 4x}{x-y} \frac{1/x}{1/x}\\
	&= \frac{y/x - 4} {1 - y/x}
	\intertext{let $v = \frac{y}{x}$ so $y = xv(x)$}
	\frac{dy}{dx} &= x\frac{dv}{dx} + v\\
	\frac{v -4}{1 - v} &= x\frac{dv}{dx} + v\\
	x\frac{dv}{dx} &= \frac{v^2 - 4}{1 - v}
\end{align*}
which is now separable!

\subs{Modeling with First Order Equations}
\begin{itemize}
\f Mixing Problems. These problems involve mixing quantities (salt, toxin, etc) into water at a certain rate, and water leaving the system at a certain rate. We usually want to find the amount of salt in the tank at time $t$. 
$$ \frac{dQ}{dt} = \text{rate in} - \text{rate out}$$
Where the rate out = $\frac{rQ}{volume}$. Usually, volume will remain constant.
\f Compound interest. Let the value of an investment = $S$ and the interest rate = $r$.
$$\frac{dS}{dt} = rS$$

\f Escape velocity. This is a more complex example and is on page 58.
\end{itemize}

\subs{Differences Between Linear and Nonlinear Equations}
\B{The Existence and Uniqueness Theorems}\\
Theorem 2.4.1\\
If the functions $p$ and $g$ are continuous on an open interval $I:\alpha < t < \beta$ containing the point $t = t_0$, then there exists a unique function $y = \phi(t)$ that satisfies the differential equation 
$$y' + p(t)y = g(t)$$
for each $t$ in $I$, and that also satisfies the initial condition
$$y(t_0) = y_0$$
where $y_0$ is an arbitrary prescribed initial value.
\B{WARNING}. This only holds for \B{linear, first order} equations!\\\\
Theorem 2.4.2\\
Let the functions $f$ and $\frac{\partial f}{\partial y}$ be continuous in some rectangle $\alpha < t < \beta$, $\gamma < y < \delta$ containing the point $(t_0, y_0)$. Then, in some interval $t_0 - h < t < t_0 + h$ contained in $\alpha < t < \beta$, there is a unique solution $y = \phi(t)$ of the initial value problem
$$y' = f(t,y) \text{ and } y(t_0) = y_0$$

\subs{Autonomous Equations and Population Dynamics}
An \B{autonomous} function is a first order differential equation in which the independent variable does not appear explicitly.
\begin{equation}
\frac{dy}{dx} = f(y)
\end{equation}

\begin{itemize}
\f Exponential Growth. A simple hypothesis about population growth: The rate of change of $y$ is proportional to the current value of $y$. $$\frac{dy}{dt} = ry$$
Here $r = $ the \B{rate of growth or decline}. Solving for the initial value $y(0) = y_0$, we get 
$$ y = y_0e^{rt}$$
\f Logistic Growth. The exponential model is too ideal. The logistic equation takes into account that the growth \emph{rate} depends on $y$. 
$$ \frac{dy}{dt} = (r - ay)y$$
or, in the equivalent form:
$$ \frac{dy}{dt} = r\left(1 - \frac{y}{K}\right)y$$
where $K = \frac{r}{a}$. $r$ is called the \B{intrinsic growth rate}, and $K$ is the \B{carrying capacity}.
\end{itemize}

\B{Equilibrium Solutions}. An equilibrium solution is one such that $\frac{dy}{dt} = 0$. E.g. in the logistic equation
\begin{align*}
	\frac{dy}{dx} &= r\left(1 - \frac{y}{K}\right)y = 0\\
		(1 - \frac{y}{K})y &= 0
\end{align*}
So when $y = 0,K$, our solution is at equilibrium. The equilibrium solutions of a general equation can be found by locating the roots of $f(y) = 0$. These are also called \B{critical points}.

\B{Stability}
\begin{itemize}
\f Asymptotically stable solution. If the $y$ tends toward the solution regardless of whether it is initially above or below it.
\f Unstable solution. The only way this can be a solution is if the initial condition is exactly this.
\f Semistable solution. If solutions on one side tend to approach it, and depart from it on the other side.
\end{itemize}

\subs{Exact Equations and Integrating Factors}
Suppose we have a differential equation of the form (4)
$$M(x,y) + N(x,y)y' = 0$$
Further suppose we can find a function $\psi$ such that 
$$\frac{\partial \psi}{\partial x}(x,y) = M(x,y)$$
$$\frac{\partial \psi}{\partial y}(x,y) = N(x,y)$$
$$\psi(x,y) = c$$

The differential equation is said to be \B{exact} if $M_y = N_x$. And if the equation is exact, then:
$$\psi_x(x,y) = M(x,y) \text{ and } \psi_y(x,y) = N(x,y)$$
Therefore, we integrate, holding y constant
\begin{align*}
\psi(x,y) &= \int \! M(x,y)dx = Q(x,y) + h(y)\\
\intertext{now, by differentiating this in terms of $y$ we can find $h$}
\psi_y(x,y) &= \frac{\partial Q}{\partial y}(x,y) + h'(y) = N(x,y)\\
h'(y) &= N(x,y) - \frac{\partial Q}{\partial y}(x,y)
\end{align*}

\B{Integrating Factors}. Sometimes you can convert an inexact equation into an exact equation by multiplying the equation by a suitable integrating factor (recall 2.1). Let $\mu$ be such a factor. Our goal is to make the equation
$$ \mu(x,y)M(x,y)dx + \mu(x,y)N(x,y)dy = 0$$
exact. It will only be exact if 
\begin{align*}
(\mu M)_y &= (\mu N)_x\\
\mu_yM + M_y\mu &= (\mu_xN + N_x\mu)\\
M\mu_y - N\mu_x + (M_y - N_x)\mu &= 0 
\end{align*}

\subs{Numerical Approximations: Euler's Method}
Euler's method involves
\begin{itemize}
\f Carrying out the linking of tangent lines in a systematic manner
\f The resulting linear function approximates the actual solution
\end{itemize}

the line tangent to the curve at $(t_0, y_0)$ is 
$$y = y_0 + f(t_0,y_0)(t - t_0)
$$